% ============================================== %
% Author: Simone Bianco
%
% Thesis Title: IDK something about proofs and circuits and protocols and P =/= NP?
% 
% Bachelor of Science in Computer Science, Sapienza University of Rome.
%
% https://github.com/Exyss/bsc-thesis/
%
% ============================================== %

% Document Class
\documentclass[binding=0.6cm, noexaminfo]{sapthesis}

% Packages
\usepackage{microtype}
\usepackage[english]{babel}
\usepackage[utf8]{inputenc}
\usepackage{hyperref}
\usepackage{listings}
\usepackage[dvipsnames]{xcolor}
\usepackage{xspace}
\usepackage{geometry} 
\usepackage{caption}  
\usepackage[edges]{forest}
\usepackage{subcaption}
\usepackage{float}    
\usepackage{lipsum}  
\usepackage{xargs}              
\usepackage{booktabs}
\usepackage{graphicx}
\usepackage{makecell}
\usepackage{enumitem}
\usepackage{booktabs}
\usepackage{tabularx}
\usepackage{longtable}
\usepackage{threeparttable}
\usepackage{amsthm}
\usepackage{amssymb}
\usepackage{cleveref}
\usepackage[Algoritmh]{algorithm}      
\usepackage{algpseudocode}
\usepackage[colorinlistoftodos,prependcaption,textsize=tiny]{todonotes}

\usepackage[
  backend=biber,
  style=alphabetic,
  sorting=anyt,
  minnames=3,
  minalphanames=3
]{biblatex}

% Commands and Settings
\newcommandx{\unsure}[2][1=]{\todo[linecolor=red,backgroundcolor=red!25,bordercolor=red,#1]{#2}}
\newcommandx{\change}[2][1=]{\todo[linecolor=blue,backgroundcolor=blue!25,bordercolor=blue,#1]{#2}}
\newcommandx{\info}[2][1=]{\todo[linecolor=OliveGreen,backgroundcolor=OliveGreen!25,bordercolor=OliveGreen,#1]{#2}}
\newcommandx{\improvement}[2][1=]{\todo[linecolor=Plum,backgroundcolor=Plum!25,bordercolor=Plum,#1]{#2}}
\newcommandx{\thiswillnotshow}[2][1=]{\todo[disable,#1]{#2}}
\newcolumntype{L}[1]{>{\raggedright\arraybackslash}p{#1}}
\newcounter{boxlblcounter}  
\newcommand{\makeboxlabel}[1]{\fbox{#1.}\hfill}
\newenvironment{boxlabel}
  {\begin{list}
    {\arabic{boxlblcounter}}
    {\usecounter{boxlblcounter}
     \setlength{\labelwidth}{3em}
     \setlength{\labelsep}{0em}
     \setlength{\itemsep}{2pt}
     \setlength{\leftmargin}{1.5cm}
     \setlength{\rightmargin}{2cm}
     \setlength{\itemindent}{0em} 
     \let\makelabel=\makeboxlabel
    }
  }
{\end{list}}
\lstdefinestyle{myStyle}{
    belowcaptionskip=1\baselineskip,
    breaklines=true,
    numberstyle=\tiny\color{gray},
    captionpos=b,
    frame=tb,
    basicstyle=\footnotesize\ttfamily,
    keywordstyle=\bfseries\color{green!40!black},
    commentstyle=\itshape\color{blue!40!black},
    identifierstyle=\color{black},
    backgroundcolor=\color{white},
}
\lstset{style=mystyle}

\setlength{\parindent}{0pt}
\setlength{\parskip}{5pt}

% Metadata
\hypersetup{pdftitle={BScThesis},pdfauthor={Simone Bianco}, urlcolor=black, linkcolor=black, colorlinks=true}
\title{[Some weird title about parity in TFNP]}
\author{Simone Bianco}
\IDnumber{1986936}
\course{Bachelor's Degree in Computer Science}
\courseorganizer{Faculty of Information Engineering, Computer Science and Statistics}
\AcademicYear{2023/2024}
\advisor{Prof. Nicola Galesi}
\coadvisor{Prof. Massimo Lauria}
\authoremail{bianco.simone@outlook.it}
\copyyear{2024}
\thesistype{Bachelor's Thesis}

% ================== CUSTOM ENVIRONMENTS ==================

\newtheorem{lemma}{Lemma}[chapter]
\newtheorem{theorem}{Theorem}[chapter]
\newtheorem{proposition}{Proposition}[chapter]
\newtheorem{corollary}{Corollary}[chapter]

\theoremstyle{definition}
\newtheorem{claimlemma}{Claim}[lemma]
\newtheorem{claimtheorem}{Claim}[theorem]
\newtheorem{definition}{Definition}[chapter]

% ================== CUSTOM MACROS ==================

\newcommand\blankchar[1][1.25ex]{\mbox{\kern.06em\vrule height1ex}\vbox{\hrule width#1}\hbox{\vrule height1ex}}         % Tape blank character

\newcommand{\vnot}{\overline}
\newcommand{\abs}[1]{\left|#1\right|}
\newcommand{\say}[1]{\flqq\textit{#1}\frqq}
\newcommand{\floor}[1]{\left \lfloor #1 \right \rfloor}
\newcommand{\ceil}[1]{\left \lceil #1 \right \rceil}
\newcommand{\abk}[1]{\left\langle#1\right\rangle}

\newcommand{\TM}{\textsf{TM}\xspace}

\newcommand{\FP}{\textsf{FP}\xspace}
\newcommand{\FNP}{\textsf{FNP}\xspace}
\newcommand{\TFNP}{\textsf{TFNP}\xspace}
\newcommand{\TFNPdt}{\mathsf{TFNP}^{dt}\xspace}

\newcommand{\N}{\mathbb{N}}
\renewcommand{\P}{\mathbb{P}}
\newcommand{\Z}{\mathbb{Z}}
\newcommand{\R}{\mathbb{R}}
\newcommand{\F}{\mathbb{F}}

\newcommand{\Res}{\mathsf{Res}}
\newcommand{\ResP}{\mathsf{Res(\oplus)}}
\newcommand{\NS}{\mathsf{NS}}
\newcommand{\FNS}{\F_2\text{-}\mathsf{NS}}
\newcommand{\enc}{\mathrm{enc}}

% ================== DOCUMENT ==================

\addbibresource{./references.bib}

\begin{document}

\lstset{language=Python}

\frontmatter

\maketitle

% ================== DEDICATION ==================
\dedication{
    TODO.
}
% ================== ABSTRACT ==================

% \begin{abstract}
    Uhhh, idk
\end{abstract}

\let\cleardoublepage\clearpage


\cleardoublepage

\tableofcontents
\let\cleardoublepage\clearpage

\hypersetup{colorlinks=true, linkcolor=blue, citecolor=red}


\mainmatter

% ================== CHAPTERS ==================
\hypersetup{colorlinks=true, linkcolor=blue, citecolor=red}

\chapter{Notes} \label{chap:introduction}

The following definitions and proofs are a reformulation of the results shown in \cite{tfnp_characterization,separations_proof_complexity}

\begin{definition}
    A total query search problem is a sequence of relations $R = \{R_i : R_i \subseteq \{0,1\}^n \times O_i, i \in [n]\}$, where $O_i$ are finite sets called outcome sets, such that $\forall x \in \{0,1\}^n$ there is a $o \in O_i$ for which $(x,o) \in R_n$.

    A total search problem $R$ is in $\TFNP^{dt}$ if its solutions are verifiable through decision trees: for each $o \in O_n$ there is a decision tree $T_o$ with $\mathrm{poly(\log n)}$-depth such that $T_o(x) = 1 \iff (x,i) \in R_n$
\end{definition}

While total search problems are formally defined as sequences $R = \{R_1, \ldots, R_n\}$, it will often make sense to speak of an individual search problem $R_i$ in the sequence. Therefore, we will slightly abuse the notation and also call $R_i$ a total search problem.

\begin{definition}
    Given $R \subseteq \{0,1\}^n \times O$ and $S \subseteq \{0,1\}^m \times O'$, a decision tree reduction from $R$ to $S$ is a set of decision trees $T_i : \{0,1\}^n \to \{0,1\}$ and $T'_o : \{0,1\}^n \to O$, for each $i \in [m]$ and each $o \in O'$, such that
    \[\forall x \in {0,1}^n \;\; ((T_1(x), \ldots, T_m(x)), o) \in S \implies (x, T'_o(x)) \in R\]
\end{definition}

To give an easier intuition of a decision tree reduction, the decision trees $T_i$ map inputs from $Q$ to $R$, while the decision trees $T'_o$ map solutions to $R$ back into solutions of $Q$.

The \textit{size} $s$ of the reduction is the number of input bits to $S$, meaning that $s = m$, while the \textit{depth} $d$ of the reduction is the maximum depth of any tree involved in the reduction
\[d := \max(\{D(T_i) : i \in [m]\} \cup \{D(T'_o) : o \in O_m\})\]

Finally, we define the \textit{complexity} of the reduction as $\log s + d$. Moreover, the denote as $R^{dt}(S)$ the minimum complexity of any decision tree reduction from $R$ to $S$.

Using the previous definition, we can define complexity classes of total query search problems via decision tree reductions. Given a total query search problem $S = (S_n)$, we define the subclass of problems reducible to $S$ as:
\[S^{dt} := \{R : S^{dt}(R) = \mathrm{poly}(\log n)\}\]
where $R = (R_n)$.

Any total query search problem  with solution verifiers $T_o$ for each $o \in O$ can be \textit{encoded} into a canonical unsatisfiable $\CNF$ formula. 

\begin{proposition}
    Given a total query search problem $R \subseteq \{0,1\}^n \times O$, there exists an unsatisfiable $\CNF$ formula $F$ defined on $\abs{O}$-many variables such that $R = \Search(F)$.
    
    This formula is called the canonical $\CNF$ encoding of $R$.
\end{proposition}

\begin{proof}
    Since $R \in \TFNP^{dt}$, for each $o \in O$ there is a $\mathrm{poly}(\log n)$-depth decision tree $T_o$ that verifies $R$. Then, for each $T_o$, let $C_o$ be the clause obtained by taking the disjunction over the conjunction of the literals along each of the accepting paths in $T_o$, meaning that $C_o$ is a $\mathsf{DNF}$ and, by De Morgan's theorem, that $\lnot{C_o}$ is a $\CNF$.

    Let $F := \bigwedge\limits_{o \in O} \lnot{C_o}$. Since each $R$ is a total search problem, for each input there is a valid output, implying that at least one tree $T_o$ will have an accepting path. Hence, by definition of $C_o$, we get that $\lnot{C_o} = 0$, implying that there will always be a false clause in $F$ and thus that $F$ is an unsatisfiable $\CNF$, concluding that:
    \[(x,o) \in R \iff T_o(x) = 1 \iff \lnot{C_o} = 0 \iff (x,o) \in \Search(F)\]
\end{proof}

This result implies that black-box $\TFNP$ is \textit{exactly} the study of the false clause search problem. Then, instead of studying a total search problem $R$, it's sufficient to study the search problem $\Search(F)$ associated with the canonical $\CNF$ encoding $F$ of $R$.

Given a proof system $P$ and an unsatisfiable $\CNF$ formula $F$, we define the \textit{complexity required by $P$ to prove $F$}, denoted with $P(F)$, as:
\[P(F) := \min(\{\deg(\Pi) + \log \mathrm{size}(\Pi) : \Pi \text{ is a $P$-proof of $F$}\})\]
where $\deg$ is the \textit{degree} of the proof, which is a measure defined by the proof system itself. For example, for the Resolution proof system the degree is defined as the \textit{width} of the proof, which is the maximum number of literals in any clause in $\Pi$.

\begin{definition}
    Given $R \in \TFNP^{dt}$, we say that $R$ \textit{characterizes} and a proof system $P$ (and that $P$ \textit{characterizes} $R$) if it holds that $R^{dt} = \{\Search(F) : P(F) = \mathrm{poly}(\log n)\}$.
\end{definition}


\newpage

Tree-like resolutions for an unsatisfiable $\CNF$ formula are strictly connected to the decision trees that solve its associated search problem. In particular, it can be proven that the smallest tree-like refutation has the exact same structure of the smallest decision tree. 

\begin{lemma} \label{lem:treeres_dt}
    \cite{treelike_res_size}
    Let $F$ be an unsatisfiable $\CNF$ formula. If there is a tree-like refutation of $F$ with structure $T$, there also exists a decision tree with structure $T$ that solves $\Search(F)$
\end{lemma}

\begin{proof}
    We procede by induction on the size $s$ of the refutation of $F$.

    Let $F = C_1 \land \ldots \land C_m$. If $s = 1$, then the refutation is made up of only one step that ends with the empty clause, implying that $\exists i \in [m]$ such that $F = C_i = \bot$. Hence, $\Search(F)$ can be solved by the decision tree made of only one vertex labeled with $i$.

    We now assume that every formula with a tree-like refutation with a structure of size $s$ there exists a decision tree with the same structure that solves the search problem associated with the formula.

    Suppose now that the size $s$ of the refutation is bigger than 1. Let $x$ be the last variable resolved by the refutation and let $T_0$ and $T_1$ be the subtrees of $T$ such that $x$ is the root of $T_0$ and $\lnot{x}$ is the root of $T_1$.

    Consider now the formulas $F{\upharpoonright_{x=0}}$ and $F{\upharpoonright_{x=1}}$, respectively corresponding the formula $F$ with the value $0$ or $1$ assigned to $x$. It's easy to see that the subtrees $T_0$ and $T_1$ are valid refutations of the formulas $F{\upharpoonright_{x=0}}$ and $F{\upharpoonright_{x=1}}$: if $b = 0$, then $x$ evaluates to $0$, otherwise if $b = 1$ then $\lnot{x}$ evaluates to 0.

    Since $T_0$ and $T_1$ have size $s-1$, by inductive hypothesis there exist two decision tree with structure $T_0$ and $T_1$ that solve $\Search(F{\upharpoonright_{x=0}})$ and $\Search(F{\upharpoonright_{x=1}})$.

    Finally, the search problem $\Search(F)$ can be solved by the decision tree that queries $x$ and proceeds with the decision tree $T_b$ based on the value $b \in \{0,1\}$ such that $x = b$.

\end{proof}

\begin{definition}
Given two rooted trees $T$ and $T'$, we say that $T$ is embeddable in $T'$ if there exists a mapping $f : V(T) \to V(T')$ such that, for any vertices $u,v \in V(T)$, if $u$ is a parent of $v$ in $T$ then $f(u)$ is an ancestor of $f(v)$ in $T'$.
\end{definition}

\begin{lemma} \label{lem:dt_treeres}
    \cite{treelike_res_size,search_problems_dt_model}
    Let $F$ be an unsatisfiable $\CNF$ formula. If there is a decision tree with structure $T$ that solves $\Search(F)$, there also exists a tree-like refutation of $F$ with structure $T'$ such that $T'$ is embeddable in $T$.
\end{lemma}

\begin{proof}

    The main idea is to associate inductively, starting from the leaves, a clause to each vertex of $T$ in order to transform $T$ in a tree-like refutation of $F$. In particular, each vertex $v$ gets associated to a clause $C(v)$ such that every input of the decision tree that reaches $v$ falsifies $C(v)$.

    Let $F = C_1 \land \ldots \land C_m$. For all $i \in [m]$, we associate the clause $C_i$ to the leaf of $T$ labeled with $i$. This constitutes our base case.

    Consider now a vertex $v$ that isn't a leaf. Let $x$ be the variable that labels $v$ and let $u_0, u_1$ be the vertices such that the edge $(v, u_0)$ is taken if $x = 0$ and the edge $(v, u_1)$ is taken if $x = 1$.  By induction, assume that $u_0$ and $u_1$ have already been associated with the clauses $C_0$ and $C_1$.

    By way of contradiction, suppose that $C_0$ contains the literal $\lnot{x}$. Then, since in a decision tree each variable can be queried only once in every path, there will always be an input with $x = 0$ that reaches $v$. Since $x = 0$ and since $C_0$ contains $\lnot{x}$, this input would satisfy $C_0$, contradicting the fact that $C_0$ was associated to $u_0$ in a way that it is falsified by every input.

    Thus, the only possibility is that $C_0$ can't contain the literal $\lnot{x}$. Similarly, we can show that $C_1$ cant' contain the literal $x$. This leaves us with only two possibilities: either $C_0 = x \lor \alpha$ and $C_1 = \lnot{x} \lor \beta$ or one of $C_0, C_1$ doesn't contain $x, \lnot{x}$.

    In the first case, we can simply associate to $v$ the clause $C = \alpha \lor \beta$.  In the second case, we associate to $v$ the clause that doesn't contain $x, \lnot{x}$ (chose any of them if both clauses do not contain $x, \lnot{x}$).

    In particular, we notice that the first case directly emulates the resolution rule, while the second case essentially represent "redundant steps". By "skipping" these redundant steps, we can obtain a tree $T'$ that is embeddable in $T$ and that contains only nodes on which the first case was applied. Finally, it's easy to deduce that the root node of $T'$ will always be associated with the empty clause $\bot$, concluding that $T'$ is the structure of a tree-like refutation of $F$.

\end{proof}

\begin{theorem}
Let $F$ be an unsatisfiable $\CNF$ formula. The smallest tree-like refutation of $F$ has size $s$ and depth $d$ if and only if the smallest decision tree solving $\Search(F)$ has size $s$ and depth $d$.
\end{theorem}

\begin{proof}
    Let $s$ and $d$ be the size and depth of the smallest tree-like refutation of $F$. Likewise, let $x$ and $y$ be the size and depth of the smallest decision tree solving $\Search(F)$.

    Then, by \Cref{lem:treeres_dt}, we know that there exists a decision tree that solved $\Search(F)$ with the same structure of the smallest refutation. Let $\alpha$ and $\beta$ be the size and depth of this decision tree. It's easy to see that $s = \alpha \geq x$ and $d = \beta \geq y$.

    Viceversa, by the \Cref{lem:dt_treeres}, we know that there exists a tree-like refutation of $F$ such that its structure is embeddable in the one of the smallest decision tree. Let $\gamma$ and $\beta$ be the size and depth of this tree-like refutation. Since the latter is embedded in the smallest decision tree, it's structure must be smaller or equal. Hence, it's easy to see that $x \geq \gamma \geq s$ and $y \geq \delta \geq d$. Thus, we can conclude that $s = x$ and $d = y$.

\end{proof}

\textbf{Note:} the theorem should be generalizable to each tree and not only for the smallest trees.

\newpage

\begin{figure}[H]
\centering

\begin{tikzpicture}[-,>=stealth,shorten >=1pt,auto,node distance=2.25cm, thick,main node/.style={scale=0.9,circle,draw,font=\sffamily\normalsize}]
    \node (1) []{$\bot$};
    
    \node (2) at ($(1)+(-2,1.75)$){};
    \node (3) at ($(1)+(2,1.75)$){$\lnot{y}$};

    \node (4) [above left of=2]{};
    \node (5) [above right of=2]{};

    \node (6) [above of=3]{$\lnot{y} \lor x$};
    \node (7) [right of=6]{$\lnot{x}$};
    
    \node (8) [above left of=6]{$x \lor \lnot{y} \lor z$};
    \node (9) [above right of=6]{$z \lor \lnot{x}$};

    \node (11) [right of=9]{$\lnot{z}$};
    \node (12) [above left of=2]{$y$};

    \node (14) at ($(8)+(-2.4,0)$){$y \lor \lnot{z}$};
    \node (15) [left of=14]{$y \lor z$};

    \path[every node/.style={font=\sffamily\small}]
        (15) edge (12)
        
        (6) edge (3)
        (7) edge (3)
        
        (8) edge (6)
        (11) edge (6)
        
        (9) edge (7)
        (11) edge (7)

        (11) edge (12)

        (12) edge (1)
        (3) edge (1)
    ;
\end{tikzpicture}

\caption{Dag-like refutation of the previous formula}
\end{figure}

\begin{figure}[H]
\centering

\begin{tikzpicture}[-,>=stealth,shorten >=1pt,auto,node distance=2.25cm, thick,main node/.style={scale=0.9,circle,draw,font=\sffamily\normalsize}]

    \node (1) []{$\bot$};
    
    \node (2) at ($(1)+(-2,1.75)$){$y$};
    \node (3) at ($(1)+(2,1.75)$){$\lnot{y}$};

    \node (4) [above left of=2]{};
    \node (5) [above right of=2]{};

    \node (6) [above of=3]{$\lnot{y} \lor z$};
    
    \node (8) [above left of=6]{$x \lor \lnot{y} \lor z$};
    \node (9) [above right of=6]{$z \lor \lnot{x}$};

    \node (11) [right of=9]{$\lnot{z}$};

    \node (14) at ($(8)+(-2.4,0)$){$y \lor \lnot{z}$};
    \node (15) [left of=14]{$y \lor z$};

    \path[every node/.style={font=\sffamily\small}]
        (14) edge (2)
        (15) edge (2)

        (8) edge (6)
        (9) edge (6)

        (6) edge (3)
        (11) edge (3)

        (2) edge (1)
        (3) edge (1)
    ;
\end{tikzpicture}

\caption{Tree-like refutation of the previous formula}
\end{figure}

\begin{figure}[H]
\centering

\begin{tikzpicture}[-,>=stealth,shorten >=1pt,auto,node distance=2.25cm, thick,main node/.style={scale=0.9,circle,draw,font=\sffamily\normalsize}]

    \node (1)[circle, draw]{$y$};
    
    \node (2) at ($(1)+(-2,-1.75)$) [circle, draw]{$z$};
    \node (3) at ($(1)+(2,-1.75)$) [circle, draw]{$z$};

    \node (4) [below left of=2]{};
    \node (5) [below right of=2]{};

    \node (6) [below of=3, circle, draw]{$x$};
    
    \node (8) [below left of=6, rectangle, draw]{$x \lor \lnot{y} \lor z$};
    \node (9) [below right of=6, rectangle, draw]{$z \lor \lnot{x}$};

    \node (11) [right of=9, rectangle, draw]{$\lnot{z}$};

    \node (14) at ($(8)+(-2.4,0)$) [rectangle, draw]{$y \lor \lnot{z}$};
    \node (15) [left of=14, rectangle, draw]{$y \lor z$};

    \path[every node/.style={font=\sffamily\small}]
        (14) edge [swap] node{1} (2)
        (15) edge node{0} (2)

        (8) edge node{0}(6)
        (9) edge node[swap]{1}(6)

        (6) edge node{0}(3)
        (11) edge [swap] node{1}(3)

        (2) edge node {0} (1)
        (3) edge node [swap] {1} (1)
    ;
\end{tikzpicture}

\caption{Decision tree for the previous formula}
\end{figure}

\cleardoublepage

\chapter{Preliminaries} \label{chap:preliminaries}

\section{Turing machines}

\section{Complexity measures}

\section{Proof complexity}

\cleardoublepage
\hypersetup{colorlinks=true, linkcolor=blue, citecolor=red}

\chapter{Black-box TFNP} \label{chap:bb-tfnp}

\section{The black-box model}

\section{Proof System Characterization}

\section{Natural Proof Systems}

\subsection{The Reflection Principle}

\subsection{Verification Procedures}

\section{The TFNP$^{dt}$ hierarchy}

\section{An in-depth analysis: FP$^{dt}$ = TreeRes}

\cleardoublepage
\hypersetup{colorlinks=true, linkcolor=blue, citecolor=red}

\chapter{Black-box TFNP} \label{chap:bb-tfnp}


\cleardoublepage

\chapter{Parity in black-box \textsf{TFNP}} \label{chap:parity-tfnp}

\section{Parity decision trees and Tree-like Res($\oplus$)}

The concept of parity is extensively studied in computer science. In our case, we are interested in exploring parity through the lens of \textit{linear forms modulo 2}, i.e being linear equations defined on $n$ variables over the algebraic field $\F_2$. In this field, each term can either be a 0 or a 1, with the defining characteristic that $1+1 = 0$.

\begin{definition}
    Given $n$ variables $x_1, \ldots, x_n$, we define a \textbf{linear form} as a linear equation over $\F_2$. In general, a linear form can be expressed as $\sum\limits_{i = 1}^n \alpha_i x_i$, where $\alpha_1, \ldots, \alpha_n \in \F_2$
\end{definition}

Intuitively, each sum in a linear form is nothing more than an application of the XOR operator: the linear form $x_1 + x_2$ is equal to 1 if and only if $x_1$ is \textit{different} from $x_2$ (i.e. if $x_1 = 1$ and $x_2 = 0$ or if $x_1 = 0$ and $x_2 = 1$). Additionally, in $\F_2$ the concepts of addition and subtraction are equivalent: since $1+1 = 0$, we easily get that $1 = -1$.

Through this properties, parity can be used to determine if two or more objects are equal or not. For example, consider the following system of linear forms:
\[\left \{ \begin{array}{l}
    x_1 + x_2 + x_3 = 1 \\
    x_1 + x_2 + x_4 = 1 \\
    x_1 + x_3 = 1
\end{array} \right .\]

By simplifying the linear system we get that:
\[\left \{ \begin{array}{l}
    x_1 + x_2 + x_3 = 1 \\
    x_1 + x_2 + x_4 = 1 \\
    x_1 + x_3 = 1
\end{array} \right . \longrightarrow
\left \{ \begin{array}{l}
    x_2 = 1\\
    x_1 + 1 + x_4 = 1 \\
    x_1 + x_3 = 1
\end{array} \right . \longrightarrow
\left \{ \begin{array}{l}
    x_2 = 1\\
    x_1 = x_4 \\
    x_1 = 1+x_3
\end{array} \right .\]

which implicitly tells us that $x_2 = 1$ and that $x_1 = x_4 \neq x_3$. 

\newpage

But what happens if we apply the concept of parity in decision trees? What if, instead of querying variables in order to know their value, we ask the parity of a set of values by querying linear forms? This idea gives rise to the extended model of \textbf{parity decision trees}.

Instead of being labeled by single variables, the nodes of a parity decision tree (PDT for short) are labeled by a linear form $f$. Each node has two outgoing edges, one labeled by $f = 0$ and the other labeled by $f = 1$. Every path from the root of the PDT to one of its nodes defines a system of linear forms given by all the labels of the edges on the path. In general, given the PDT $T$ and a node $v$, we denote this system with $\Phi_v^T$. Given an assignment $\alpha(x_1, \ldots, x_n)$, the output of a PDT is dictated by the parity queries made by each node.

\begin{figure}[H]
    \centering

    \begin{tikzpicture}[->,>=stealth,shorten >=1pt,auto,node distance=1.75cm, thick,main node/.style={scale=0.9,circle,draw,font=\sffamily\normalsize}]

        \node[ellipse, draw] (1)[] {$x_1+x_2+x_3$};

        \node[ellipse, draw] (2) [below left of=1, xshift=-50, ]{$x_1+x_2$};
        \node[ellipse, draw] (3) [below right of=1, xshift=50, ]{$x_1$};

        \node[ellipse, draw] (4) [below left of=2, xshift=-10, ]{$x_2+x_3$};
        \node[ellipse, draw] (5) [below right of=2, xshift=10, ]{$x_2$};
        \node[ellipse, draw] (6) [below left of=3, ]{$x_2$};
        \node[rectangle, draw] (7) [below right of=3]{$o_7$};

        \node[rectangle, draw] (8) [below left of=4, xshift=10, yshift=-10]{$o_1$};
        \node[rectangle, draw] (9) [below right of=4, xshift=-10, yshift=-10]{$o_2$};
        \node[rectangle, draw] (10) [below left of=5, xshift=10, yshift=-10]{$o_3$};
        \node[rectangle, draw] (11) [below right of=5, xshift=-10, yshift=-10]{$o_4$};
        \node[rectangle, draw] (12) [below left of=6, xshift=10, yshift=-10]{$o_5$};
        \node[rectangle, draw] (13) [below right of=6, xshift=-10, yshift=-10]{$o_6$};

        \path[every node/.style={font=\sffamily\small}]
            (1) edge[swap, color=Green]  node{0} (2)
            (1) edge node{1}(3)

            (2) edge[swap]  node{0} (4)
            (2) edge[color=Green]  node{1}(5)

            (3) edge[swap]  node{0} (6)
            (3) edge  node{1}(7)
            
            (4) edge[swap]  node{0} (8)
            (4) edge  node{1}(9)

            (5) edge[swap]  node{0} (10)
            (5) edge[color=Green]  node{1}(11)

            (6) edge[swap]  node{0} (12)
            (6) edge  node{1}(13)
        ;
    \end{tikzpicture}

    \caption{An example of parity decision tree of size 13 and depth 3.}
\end{figure}

In the above example, the green path defines the following system of linear forms:
\[\left \{ \begin{array}{l}
    x_1 + x_2 + x_3 = 0 \\
    x_1 + x_2 = 1 \\
    x_2 = 1
\end{array}\right .\]

which once simplified corresponds to the assignment $x_0 = 0, x_2 = 1, x_3 = 1$. Since a system of linear forms can have multiple solutions, many assignments could actually be mapped to the same output. However, some systems could also be unsatisfiable, meaning that the node cannot be reached by any assignment. When this happens we say that the node is \textbf{degenerate}.

Intuitively, every PDT can be converted into a normal decision tree simply by "splitting" each linear query in more queries, a process that exponentially increases the size of the tree. In fact, PDTs tend generally to be more compact that normal decision trees, even though this isn't usually true for simple problems. We define the class $\mathsf{FP}^{pdt}$ as the set of $\mathsf{TFNP}^{dt}$ problems that are efficiently solvable by a PDT.

\begin{definition}
    We define $\mathsf{FP}^{pdt}$ as the set of query search problems $R = (R_n)_{n \in \N}$ for which there exists a polylogarithmic depth PDT $T_n$ such that $T_n(x) = y$ if and only if $(x,y) \in R_n$.
\end{definition}

\newpage

Like normal decision trees, PDTs can be used to solve the false clause search problem associated with any unsatisfiable CNF. A parity decision tree for a CNF formula $F$ is a PDT defined on the same variables of $F$ where for each leaf $v$ one of the following conditions holds:
\begin{enumerate}[itemsep=0em]
    \item The leaf is \textit{degenerate}
    \item The leaf \textit{refutes} a clause $C$ of $F$, meaning that the system $\Phi_v^T$ is satisfiable and every one of its solutions falsifies $C$
    \item The leaf \textit{satisfies} a clause $C$ of $F$, meaning that the system $\Phi_v^T$ has only one solution and it also satisfies $C$
\end{enumerate}

We observe that if a node doesn't meet any of these conditions then it cannot be a leaf node. Moreover, we also observe that the system associated with the root of any PDT is always satisfiable due to it containing no linear forms.

Since we are interested in studying PDTs for refusing unsatisfiable CNF formulas, the third case will never be true for any leaf. However, we still need a way to exclude the first case, since an unsatisfiable system cannot be associated with any assignment. Luckily, each degenerate PDT can be conveniently converted into a non-degenerate one through a very simple process \cite{res_lin_2}.

\begin{proposition}
    \label{degenerate}
    Let $F$ be an unsatisfiable CNF formula. If $\mathrm{Search}(F)$ can be solved with a degenerate PDT of size $s$ and depth $d$, it can also be solved with a non-degenerate PDT of size at most $s$ and depth at most $d$.
\end{proposition}

\begin{proof}
    
    Let $T$ be a degenerate PDT of size $s$ and depth $d$ that solves $\mathrm{Search}(F)$. Let $U$ be the set of degenerate nodes of $T$. Notice that since $\Phi_r^T$ is empty, thus always satisfiable, we know that $r \notin U$.

    Consider the node $u \in U$ with the minimal distance from the root $r$. Since $u$ is not the root of $T$, there must be two vertices $p$ and $s$ such that $p$ is the parent of $u$ and $s$ is the sibling of $u$.

    We notice that $\Phi_s^T$ must be satisfiable: if this wasn't true then both $\Phi_s^T$ and $\Phi_u^T$ would be unsatisfiable, which can only be true if $\Phi_p^T$ is also unsatisfiable, but we chose $w$ as the node in $U$ with minimal distance. Since $\Phi_s^T$ is satisfiable, the label $f = \alpha$ on the edge $(p,s)$ must be already contained inside the system $\Phi_p^T$, meaning that each assignment that satisfies $\Phi_p^T$ also satisfies $\Phi_s^T$.

    We construct a new PDT $T'$ by removing the subtree $T_u$ with root $u$ from the initial PDT $T$ and by contracting the edge $(p,s)$, merging the two the nodes $p$ and $s$ into a single node $v$. In other words, the subtree $T_u$ gets removed and the children of $s$ become the new children of $p$. Each assignment that satisfies $\Phi_p^T$ also satisfies $\Phi_v^{T'}$, concluding that $T'$ also solves $\mathrm{Search}(F)$.

    By repeating the process until $U$ is empty, we get a non-degenerate PDT that solves $\mathrm{Search}(F)$ of size at most $s$ and depth at most $d$.
\end{proof}

\newpage

Now, we are interested in finding a canonical proof system that can characterize our brand new class of problems. Consider a generic system of linear forms $\Phi$. This system can be viewed as the conjunction of the linear forms that it describes:
\[\left \{ \begin{array}{l}
    f_1 = \alpha_1 \\
    f_2 = \alpha_2 \\
    \vdots \\
    f_k = \alpha_k
\end{array}\right . \iff (f_1 = \alpha_1) \land (f_2 = \alpha_2) \land \ldots \land (f_k = \alpha_k)\]

We can rewrite these conjunctions as a negation of a disjunction:
\[\bigwedge_{i = 1}^k (f_i = \alpha_i) \iff \lnot \bigvee_{i = 1}^k \lnot (f_i = \alpha_i) \iff \lnot \bigvee_{i = 1}^k (f_i = 1 + \alpha_i)\]

which implies that the negation of the system is equivalent to a set of disjunctions:
\[\lnot \bigwedge_{i = 0}^k (f_1 = \alpha_1) \iff \bigvee_{i = 0}^k (f_1 = 1 + \alpha_1)\]

We say that such set of disjunction is a \textbf{linear clause}. More generally, a \textit{linear CNF formula} is a conjunction of linear clauses.

\begin{definition}
    A linear CNF formula is a conjunction of $m$ disjunctions of linear equations over $\F_2$.
    \[\bigwedge_{i = 1}^m \bigvee_{j = 1}^{k_i} (f_j = \alpha_j)\]
\end{definition}

Generally, linear CNF formulas can assume a complex structure, such as the following:
\[((x_1+x_2 = 0) \lor (x_1 = 1)) \land ((x_2 + x_3 + x_4 = 1) \lor (x_2 + x_4 = 0))\]

Moreover, any standard CNF formula can be described as a linear CNF formula simply by treating each clause as a disjunction of linear forms made of a single term. For example, the CNF $(x_1 \lor \vnot{x_2}) \land (\vnot{x_3} + x_1)$ can be written as the following linear CNF formula:
\[((x_1 = 1) \lor (x_2 = 0)) \land ((x_3 = 0) \lor (x_1 = 1))\]

We call this the \textit{linear encoding} of a CNF. Once we have defined a way to treat CNF formulas as linear forms, we are now ready to define a new proof system. We define the \textbf{parity resolution} proof system, noted with $\ResP$, by the following two rules:
\begin{itemize}
    \item \textit{Cut}: given two linear clauses $(f = 0) \lor C$ and $(f = 1) \lor D$, we can derive the linear clause $C \lor D$
    \item \textit{Weakening}: given a linear clause $C$, we can derive any linear clause $D$ such that $C \implies D$.
\end{itemize}

Similarly to normal resolution, in $\ResP$ any derivation of a linear clause $C$ from a linear CNF $F$ is a sequence of linear clauses that ends with $C$, where every clause is either an axiom of $F$ or it can be derived from previous clauses through one of the two derivation rules. A linear CNF is unsatisfiable if and only if the empty linear clause can be derived from it.
Furthermore, each clause in a derivation is used at most once we say that the derivation has a \textit{tree-like} structure.

By definition, the weakening rule makes this proof system powerful since semantical implications can be used in many forms. For example, consider the following linear CNF:
\[(x = 1) \land (x+y = 1) \land ((x = 0) \lor (y = 1))\]

By rewriting the last linear clause as negation of a conjunction, we notice that:
\[(x = 0) \lor (y = 1) \equiv \lnot ((x = 1) \land (y = 0))\]

By simple substitution we get that:
\[\lnot ((x = 1) \land (y = 0)) \implies  \lnot ((x = 1) \land (x+y = 1))\]

which is equivalent to:
\[\lnot ((x = 1) \land (x+y = 1)) \equiv  (x = 0) \lor (x+y = 0)\]

concluding that $(x = 0) \lor (y = 1) \models (x = 0) \lor (x+y = 0)$. Proceeding with the cut rule, we get the following tree-like refutation:
\begin{figure}[H]
    \centering
    
    \begin{tikzpicture}[-,>=stealth,shorten >=1pt,auto,node distance=2.25cm, thick,main node/.style={scale=0.9,circle,draw,font=\sffamily\normalsize}]
    
        \node (1) []{$\bot$};
        
        \node (3) [above right of=1]{$(x = 0)$};
    
        \node (6) [above right of=3]{$(x=0) \lor (x+y = 0)$};
        
        \node (8) [above of=6]{$(x = 0) \lor (y = 1)$};
    
        \node (14) at ($(8)+(-4,0)$){$(x+y=1)$};
        \node (15) [left of=14]{$(x=1)$};
    
        \path[every node/.style={font=\sffamily\small}]
            (14) edge (3)
            (15) edge (1)
    
            (8) edge (6)
    
            (6) edge (3)
    
            (3) edge (1)
        ;
    \end{tikzpicture}
\end{figure}

Parity decision trees and tree-like parity resolution can be viewed as two sides of the same coin. In fact, any PDT can be used to construct a tree-like $\ResP$ refutation and vice versa while maintaining (almost) the same size and depth \cref{res_lin_2}.

\begin{lemma}
    Let $F$ be an unsatisfiable linear CNF formula. If there is a PDT that solves $\mathrm{Search}(F)$ of size $s$ and depth $d$, there also exists a tree-like $\ResP$ refutation of $F$ of size at most $2s$, depth at most $d+1$ and weakening rule applied only to the leaves.
\end{lemma}

\begin{proof}
    Let $T$ be a PDT of size $s$ and depth $d$ that solved $\mathrm{Search}(F)$. By \Cref{degenerate}, we assume that $T$ is non-degenerate. We label every node $v$ of $T$ with the negation of its associated linear system. In other words, every node $v$ is labeled with the linear clause $\lnot \Phi_v^T$. Clearly, every node is a result of the cut rule being applied on it's children, where the root node is the empty clause.

    Since $T$ is a PDT that solves $\mathrm{Search}(F)$, each leaf refutes a linear clause of $F$. Hence, for each leaf $u$ we have that $\Phi_{u}^T \implies \lnot C$ for some linear clause $C$ of $F$, which equivalently means that $C \implies \lnot \Phi_u^T$, concluding that the linear clause of each leaf is actually a weakening of a clause of $F$.
    
    Then, for each leaf $u$ we can add a new neighbor node $w$ and label it with the clause $C$, where the edge $(w,u)$ becomes an application of the weakening rule. This process increases the depth of the tree by 1 and increases the size by at most $s$.

\end{proof}


\begin{lemma}
    Let $F$ be an unsatisfiable linear CNF formula. If there is a tree-like $\ResP$ refutation of $F$ of size $s$ and depth $d$, there also exists a PDT that solves $\mathrm{Search}(F)$ of size at most $s$ and depth at most $d$.
\end{lemma}

\begin{proof}
    Let $T$ be the proof tree that refutes $F$. We label each edge of $T$ whose associated clauses involve a cut rule, while all the other weakening edges remain unlabeled. In particular, if a resolution rule is applied to the clauses $(f = 0) \lor D_1$ and $(f = 1) \lor D_2$ obtaining the clause $D_1 \lor D_2$, we label the edge from the first to the third with $f = 1$, while the other edge is labeled with $f = 0$.

    By induction on the depth of a vertex of $T$, we show that the clause written in $v$ contradicts the system $\Phi_v^T$. The root node contains the empty clause and is labeled by an empty system, making the statement trivially true. Assume now that the statement holds for a generic node $v$. We have to show that the hypothesis also holds for its children $u$ and $w$.

    Suppose that $v$ is the result of a cut rule application, where $D_1 \lor D_2$ is the clause inside $v$. Assume that $u$ is the node that contains $(f = 0) \lor D_1$ while $w$ contains $(f = 1) \lor D_2$. By inductive hypothesis, we know that $D_1 \lor D_2$ contradicts the system $\Phi_v^T$ and equivalently that the system $\lnot (\lnot D_1 \land \lnot D_2)$ contradicts $\Phi_v^T$. This means that se of equalities in $D_1$ contradict $\Phi_v^T$. Moreover, we know that $\Phi_u^T = \Phi_v^T \land (f = 1)$, concluding that $(f = 0) \lor D_1$ contradicts $\Phi_u^T$. Likewise, we can show that $(f = 1) \lor D_2$ contradicts $\Phi_w^T$. Suppose now that $v$ is the result of a weakening rule, where $u$ is the only child. Since $(v,u)$ is unlabeled, we get that $\Phi_v^T = \Phi_u^T$. Furthermore, since $v$ is the result of a weakening applied to $u$, we know that the clause in $u$ semantically implies the clause in $v$, but by inductive hypothesis we know that the clause in $v$ contradicts the system $\Phi_v^T$, meaning that $u$ must also contradict the system $\Phi_v^T = \Phi_u^T$. Finally, if $v$ is a leaf then the statement is trivially true since it refutes a clause of $F$.

    By contracting all the unlabeled edges given by the weakening rules, we get a parity decision tree that solves $\mathrm{Search}(F)$. Due to this final step, the size of the PDT is at most $s$ and its depth is at most $d$. 

\end{proof}

By these two lemmas, it's easy to see that tree-like parity resolution is a proof system capable of characterizing the class $\mathsf{FP}^{pdt}$.

\begin{corollary}
    $\mathsf{FP}^{pdt}(\mathrm{Search}(F)) = \Theta(\mathsf{TreeRes}(\oplus)(F))$
\end{corollary}


\section{Nullstallensatz over $\F_2$}

\section{$\mathrm{NS-}\F_2$ simulates $\mathrm{TreeRes}(\oplus)$}

\cleardoublepage

% % ================== Notes ==================
% 

\chapter{Notes} \label{chap:notes}

\section{Treelike $\Res$ and Nullstellensatz}

\begin{definition}[$\FNS$ encoding of $\Res$]
    Given a $\Res$ linear clause $C = \bigvee\limits_{i = 0}^{k_1} x_i \lor  \bigvee\limits_{j = 0}^{k_2} \overline{x_j}$, the $\FNS$ encoding of $C$ is defined as $\enc(C) := \prod\limits_{i = 0}^{k_1} x_i \cdot \prod\limits_{j = 0}^{k_2} (1-x_j)$.
    
    In general, a $\ResP$ formula $F = C_1 \land \ldots \land C_m$ defined on the variables $x_1, \ldots, x_n$ gets encoded in $\FNS$ as the set of axioms $P_F = \{\enc(C_i) = 0 \mid 1 \leq i \leq m\} \cup \{x_j^2-x_j = 0 \mid 1 \leq j \leq n\}$.
\end{definition}

\begin{theorem}
    Let $F$ be an unsatisfiable CNF. If $T$ is $\ResP$ refutation of $F$ of size $s$ then there is $\NS$ refutation of $F$ of degree $O(\log(s))$.
\end{theorem}

\begin{proof}
    Let $F = C_1 \land \cdots \land C_n$. We proceed by strong induction on the size $s$.
    
    If $s = 1$ then the $T$ contains only the empty clause $\bot$, meaning that it also is one of the starting clauses and thus one of the axioms. We notice that $\mathrm{enc}(\bot) = 1$, which easily concludes that $\bot \vdash_{0}^\NS 1$.

    Suppose now that $s > 1$. Let $\mathcal{L}$ be axioms of $T$. Since $T$ is a binary tree, by \Cref{13_23_lewis} we know that there is a clause $C_k$, i.e. a node, of $T$ such that $T_{C_k}$ has size between $\floor{\frac{1}{3} s}$ and $\ceil{\frac{2}{3} s}$.

    Let $T' = (T - T_{C_k}) \cup \{C_k\}$. Due to the size of $T_{C_k}$, we get that $T'$ has size between $\floor{\frac{1}{3} s}+1$ and $\ceil{\frac{2}{3} s}+1$. Moreover, we notice that since $T$ is a treelike refutation it holds that $T_{C_k}$ and $T'$ work with different clauses (except $C_k$), thus their axioms are disjoint. Let $\mathcal{L}_1, \mathcal{L}_2$ be the two sets of axioms respectively used by $T_{C_k}$ and $T'$.
    
    By construction, we notice that $T_{C_k}$ derives the clause $C_k$ using the axioms $\mathcal{L}_1$, while $T_{C_k}$ derives the clause $\bot$ using the axioms $\mathcal{L}_2, C_k$. Thus, since $T_{C_k}$ and $T'$ have size lower than $s$, by induction hypothesis we get that $\enc(\mathcal{L}_1) \vdash_{c_1 \cdot \log s}^\NS \enc(C_k)$ and $\enc(\mathcal{L}_2), \enc(C_k) \vdash_{c_2 \cdot \log s}^\NS 1$ for some constants $c_1, c_2$. By \Cref{neg_refutation} we easily conclude that $\enc(\mathcal{L}_1), (1- \enc(C_k)) \vdash_{c_1 \cdot \log s}^\NS 1$ and, by \Cref{union_ref}, that $\enc(\mathcal{L}_1), \enc(\mathcal{L}_2) \vdash_{(c_1+c_2) \cdot \log s}^\NS 1$. Finally, since $\mathcal{L}_1 \cup \mathcal{L}_2 = \mathcal{L}$, we get that $\enc(\mathcal{L}) \vdash_{(c_1+c_2) \cdot \log s}^\NS 1$, meaning that $\mathcal{L}$ has a $\NS$ refutation of degree $O(\log s)$.

\end{proof}

\section{Treelike $\ResP$ and Nullstellensatz}

\begin{definition}[$\FNS$ encoding of $\Res(\oplus)$]
    Given a $\ResP$ linear clause $C = \bigvee\limits_{i = 0}^k (\ell_i = \alpha_i)$, the $\FNS$ encoding of $C$ is defined as $\enc_{\oplus}(C) := \prod\limits_{i = 0}^k (\alpha - \ell_i)$.
    
    In general, a $\ResP$ formula $F = C_1 \land \ldots \land C_m$ defined on the variables $x_1, \ldots, x_n$ gets encoded in $\FNS$ as the set of axioms $P_F = \{\enc_{\oplus}(C_i) = 0 \mid 1 \leq i \leq m\} \cup \{x_j^2-x_j = 0 \mid 1 \leq j \leq n\}$.
\end{definition}

\begin{theorem}[\cite{res_parity}]
    \;
    \begin{enumerate}
        \item Every tree-like $\ResP$ proof of an unsatisfiable formula $F$ may be translated to a parity decision tree for $F$ without increasing the size of the tree.
        
        \item Every parity decision tree for an unsatisfiable linear CNF may be translated into a tree-like $\ResP$ proof and the size of the resulting proof is at most twice the size of the parity decision tree (and where the weakening is applied only to the axioms).
    \end{enumerate}
\end{theorem}

\begin{corollary}
    Every tree-like $\ResP$ proof of an unsatisfiable formula $F$ can be converted to a tree-like $\ResP$ proof of at most double the size and with weakening applied only to the axioms.
\end{corollary}


\cleardoublepage

% ================== ACKNOWLEDGMENTS ==================
\chapter*{Acknowledgements}

\addcontentsline{toc}{chapter}{Acknowledgements}

No idea

\backmatter
\phantomsection

% ================== BIBLIOGRAPHY ==================
\addcontentsline{toc}{chapter}{Bibliography}
% \bibliographystyle{sapthesis}
% \bibliography{./references.bib}
\printbibliography

\end{document}