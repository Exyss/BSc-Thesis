

\chapter{Notes} \label{chap:notes}

Tree-like resolutions for an unsatisfiable CNF formula are strictly connected to the decision trees that solve its associated search problem. In particular, it can be proven that the smallest tree-like refutation has the exact same structure of the smallest decision tree. 

\begin{lemma} \label{lem:treeres_dt}
    \cite{treelike_res_size}
    Let $F$ be an unsatisfiable CNF formula. If there is a tree-like refutation of $F$ with structure $T$, there also exists a decision tree with structure $T$ that solves $\mathrm{Search}(F)$
\end{lemma}

\begin{proof}
    We procede by induction on the size $s$ of the refutation of $F$.

    Let $F = C_1 \land \ldots \land C_m$. If $s = 1$, then the refutation is made up of only one step that ends with the empty clause, implying that $\exists i \in [m]$ such that $F = C_i = \bot$. Hence, $\mathrm{Search}(F)$ can be solved by the decision tree made of only one vertex labeled with $i$.

    We now assume that every formula with a tree-like refutation with a structure of size $s$ there exists a decision tree with the same structure that solves the search problem associated with the formula.

    Suppose now that the size $s$ of the refutation is bigger than 1. Let $x$ be the last variable resolved by the refutation and let $T_0$ and $T_1$ be the subtrees of $T$ such that $x$ is the root of $T_0$ and $\lnot{x}$ is the root of $T_1$.

    Consider now the formulas $F{\upharpoonright_{x=0}}$ and $F{\upharpoonright_{x=1}}$, respectively corresponding the formula $F$ with the value $0$ or $1$ assigned to $x$. It's easy to see that the subtrees $T_0$ and $T_1$ are valid refutations of the formulas $F{\upharpoonright_{x=0}}$ and $F{\upharpoonright_{x=1}}$: if $b = 0$, then $x$ evaluates to $0$, otherwise if $b = 1$ then $\lnot{x}$ evaluates to 0.

    Since $T_0$ and $T_1$ have size $s-1$, by inductive hypothesis there exist two decision tree with structure $T_0$ and $T_1$ that solve $\mathrm{Search}(F{\upharpoonright_{x=0}})$ and $\mathrm{Search}(F{\upharpoonright_{x=1}})$.

    Finally, the search problem $\mathrm{Search}(F)$ can be solved by the decision tree that queries $x$ and proceeds with the decision tree $T_b$ based on the value $b \in \{0,1\}$ such that $x = b$.

\end{proof}

\begin{definition}
Given two rooted trees $T$ and $T'$, we say that $T$ is embeddable in $T'$ if there exists a mapping $f : V(T) \to V(T')$ such that, for any vertices $u,v \in V(T)$, if $u$ is a parent of $v$ in $T$ then $f(u)$ is an ancestor of $f(v)$ in $T'$.
\end{definition}

\begin{lemma} \label{lem:dt_treeres}
    \cite{treelike_res_size,search_problems_dt_model}
    Let $F$ be an unsatisfiable CNF formula. If there is a decision tree with structure $T$ that solves $\mathrm{Search}(F)$, there also exists a tree-like refutation of $F$ with structure $T'$ such that $T'$ is embeddable in $T$.
\end{lemma}

\begin{proof}

    The main idea is to associate inductively, starting from the leaves, a clause to each vertex of $T$ in order to transform $T$ in a tree-like refutation of $F$. In particular, each vertex $v$ gets associated to a clause $C(v)$ such that every input of the decision tree that reaches $v$ falsifies $C(v)$.

    Let $F = C_1 \land \ldots \land C_m$. For all $i \in [m]$, we associate the clause $C_i$ to the leaf of $T$ labeled with $i$. This constitutes our base case.

    Consider now a vertex $v$ that isn't a leaf. Let $x$ be the variable that labels $v$ and let $u_0, u_1$ be the vertices such that the edge $(v, u_0)$ is taken if $x = 0$ and the edge $(v, u_1)$ is taken if $x = 1$.  By induction, assume that $u_0$ and $u_1$ have already been associated with the clauses $C_0$ and $C_1$.

    By way of contradiction, suppose that $C_0$ contains the literal $\lnot{x}$. Then, since in a decision tree each variable can be queried only once in every path, there will always be an input with $x = 0$ that reaches $v$. Since $x = 0$ and since $C_0$ contains $\lnot{x}$, this input would satisfy $C_0$, contradicting the fact that $C_0$ was associated to $u_0$ in a way that it is falsified by every input.

    Thus, the only possibility is that $C_0$ can't contain the literal $\lnot{x}$. Similarly, we can show that $C_1$ can't contain the literal $x$. This leaves us with only two possibilities: either $C_0 = x \lor \alpha$ and $C_1 = \lnot{x} \lor \beta$ or one of $C_0, C_1$ doesn't contain $x, \lnot{x}$.

    In the first case, we can simply associate to $v$ the clause $C = \alpha \lor \beta$.  In the second case, we associate to $v$ the clause that doesn't contain $x, \lnot{x}$ (chose any of them if both clauses do not contain $x, \lnot{x}$).

    In particular, we notice that the first case directly emulates the resolution rule, while the second case essentially represent "redundant steps". By "skipping" these redundant steps, we can obtain a tree $T'$ that is embeddable in $T$ and that contains only nodes on which the first case was applied. Finally, it's easy to deduce that the root node of $T'$ will always be associated with the empty clause $\bot$, concluding that $T'$ is the structure of a tree-like refutation of $F$.

\end{proof}

\begin{theorem}
Let $F$ be an unsatisfiable CNF formula. The smallest tree-like refutation of $F$ has size $s$ and depth $d$ if and only if the smallest decision tree solving $\mathrm{Search}(F)$ has size $s$ and depth $d$.
\end{theorem}

\begin{proof}
    Let $s$ and $d$ be the size and depth of the smallest tree-like refutation of $F$. Likewise, let $x$ and $y$ be the size and depth of the smallest decision tree solving $\mathrm{Search}(F)$.

    Then, by \Cref{lem:treeres_dt}, we know that there exists a decision tree that solved $\mathrm{Search}(F)$ with the same structure of the smallest refutation. Let $\alpha$ and $\beta$ be the size and depth of this decision tree. It's easy to see that $s = \alpha \geq x$ and $d = \beta \geq y$.

    Viceversa, by \Cref{lem:dt_treeres}, we know that there exists a tree-like refutation of $F$ such that its structure is embeddable in the one of the smallest decision tree. Let $\gamma$ and $\beta$ be the size and depth of this tree-like refutation. Since the latter is embedded in the smallest decision tree, it's structure must be smaller or equal. Hence, it's easy to see that $x \geq \gamma \geq s$ and $y \geq \delta \geq d$. Thus, we can conclude that $s = x$ and $d = y$.

\end{proof}

\textbf{Note:} \cite{proofs_circuits_communication} says that this theorem should be generalizable to each tree and not only for the smallest trees (doubt this is true)

\newpage

\begin{figure}[H]
\centering

\begin{tikzpicture}[-,>=stealth,shorten >=1pt,auto,node distance=2.25cm, thick,main node/.style={scale=0.9,circle,draw,font=\sffamily\normalsize}]
    \node (1) []{$\bot$};
    
    \node (2) at ($(1)+(-2,1.75)$){};
    \node (3) at ($(1)+(2,1.75)$){$\lnot{y}$};

    \node (4) [above left of=2]{};
    \node (5) [above right of=2]{};

    \node (6) [above of=3]{$\lnot{y} \lor x$};
    \node (7) [right of=6]{$\lnot{x}$};
    
    \node (8) [above left of=6]{$x \lor \lnot{y} \lor z$};
    \node (9) [above right of=6]{$z \lor \lnot{x}$};

    \node (11) [right of=9]{$\lnot{z}$};
    \node (12) [above left of=2]{$y$};

    \node (14) at ($(8)+(-2.4,0)$){$y \lor \lnot{z}$};
    \node (15) [left of=14]{$y \lor z$};

    \path[every node/.style={font=\sffamily\small}]
        (15) edge (12)
        
        (6) edge (3)
        (7) edge (3)
        
        (8) edge (6)
        (11) edge (6)
        
        (9) edge (7)
        (11) edge (7)

        (11) edge (12)

        (12) edge (1)
        (3) edge (1)
    ;
\end{tikzpicture}

\caption{Dag-like refutation of the previous formula}
\end{figure}

\begin{figure}[H]
\centering

\begin{tikzpicture}[-,>=stealth,shorten >=1pt,auto,node distance=2.25cm, thick,main node/.style={scale=0.9,circle,draw,font=\sffamily\normalsize}]

    \node (1) []{$\bot$};
    
    \node (2) at ($(1)+(-2,1.75)$){$y$};
    \node (3) at ($(1)+(2,1.75)$){$\lnot{y}$};

    \node (4) [above left of=2]{};
    \node (5) [above right of=2]{};

    \node (6) [above of=3]{$\lnot{y} \lor z$};
    
    \node (8) [above left of=6]{$x \lor \lnot{y} \lor z$};
    \node (9) [above right of=6]{$z \lor \lnot{x}$};

    \node (11) [right of=9]{$\lnot{z}$};

    \node (14) at ($(8)+(-2.4,0)$){$y \lor \lnot{z}$};
    \node (15) [left of=14]{$y \lor z$};

    \path[every node/.style={font=\sffamily\small}]
        (14) edge (2)
        (15) edge (2)

        (8) edge (6)
        (9) edge (6)

        (6) edge (3)
        (11) edge (3)

        (2) edge (1)
        (3) edge (1)
    ;
\end{tikzpicture}

\caption{Tree-like refutation of the previous formula}
\end{figure}

\begin{figure}[H]
\centering

\begin{tikzpicture}[-,>=stealth,shorten >=1pt,auto,node distance=2.25cm, thick,main node/.style={scale=0.9,circle,draw,font=\sffamily\normalsize}]

    \node (1)[circle, draw]{$y$};
    
    \node (2) at ($(1)+(-2,-1.75)$) [circle, draw]{$z$};
    \node (3) at ($(1)+(2,-1.75)$) [circle, draw]{$z$};

    \node (4) [below left of=2]{};
    \node (5) [below right of=2]{};

    \node (6) [below of=3, circle, draw]{$x$};
    
    \node (8) [below left of=6, rectangle, draw]{$x \lor \lnot{y} \lor z$};
    \node (9) [below right of=6, rectangle, draw]{$z \lor \lnot{x}$};

    \node (11) [right of=9, rectangle, draw]{$\lnot{z}$};

    \node (14) at ($(8)+(-2.4,0)$) [rectangle, draw]{$y \lor \lnot{z}$};
    \node (15) [left of=14, rectangle, draw]{$y \lor z$};

    \path[every node/.style={font=\sffamily\small}]
        (14) edge [swap] node{1} (2)
        (15) edge node{0} (2)

        (8) edge node{0}(6)
        (9) edge node[swap]{1}(6)

        (6) edge node{0}(3)
        (11) edge [swap] node{1}(3)

        (2) edge node {0} (1)
        (3) edge node [swap] {1} (1)
    ;
\end{tikzpicture}

\caption{Decision tree for the previous formula}
\end{figure}


\section{The $\frac{1}{3}, \frac{2}{3}$ lemma}

\begin{definition}
    Given a tree $T$ and a node $v$, we denote as $T_v$ the subtree of $T$ having $v$ as its radix. 
\end{definition}

\begin{lemma}[Lewis' $\frac{1}{3}, \frac{2}{3}$ lemma \cite{1-3_2-3}]
    \label{13_23_lewis}
    If $T$ is a binary tree of size $s > 1$ then there is a node $v$ such that the subtree $T_v$ has size between $\floor{\frac{1}{3} s}$ and $\ceil{\frac{2}{3} s}$.
\end{lemma}

\begin{proof}
    Let $r$ be the radix of $T$ and let $\ell$ be a leaf of $T$ with the longest possible path $r \to \ell$. Let $v_1, \ldots, v_k$ be the nodes of such path, where $r = v_1$ and $\ell = v_k$. For each index $i$ such that $1 \leq i \leq k$, let $a_i b_i$ be the two children of $v_i$.

    \begin{claimlemma}
        For any index $i$, if $T_{v_i}$ has size at least $\floor{\frac{1}{3} s}$ then for some index $j$, where $i \leq j \leq k$, it holds that $T_{v_j}$ has size between $\floor{\frac{1}{3} s}$ and $\ceil{\frac{2}{3} s}$.
    \end{claimlemma}

    \begin{proof}[Proof of the claim]
        If $T_{v_i}$ has also size less than $\ceil{\frac{2}{3} s}$ then we are done. Otherwise, since $T_{v_i} = \{v_i\} \cup T_{a_i} \cup T_{b_i}$, one between the subtrees $T_{a_i}, T_{b_i}$ must have size at least $\frac{1}{2} \ceil{2}{3} s - 1$, meaning that it has size at least $\floor{\frac{1}{3} s}$. If this subtree has also a size at most $\ceil{\frac{2}{3} s}$ then we are done. Instead, if this doesn't hold for both subtrees, we can repeat the process (assuming that $v_{i+1} := a_i$ without loss of generality) since we know that $T_{v_{i+1}}$ has size greater than $\floor{\frac{1}{3} s}$.

        By way of contradiction, suppose that this process never finds a subtree with size at most $\ceil{\frac{2}{3} s}$. Then, this would mean that it also holds for $v_k = \ell$. However, since $\ell$ is a leaf, we know that $T_{v_\ell}$ must have size 1, which is definitely at most $\ceil{\frac{2}{3} s}$ for any value of $s$, giving a contradiction. Thus, there must be a node that terminates the process.

    \end{proof}
    
    Since $T_{v_1} = \{r\} \cup T_{a_1} \cup T_{b_1}$, we know that for both of these subtrees must have at least $\floor{\frac{1}{3} s}$. Thus, assuming that $a_1 = v_{2}$, the claim directly concludes the proof.
    
\end{proof}

\section{Nullstellensatz}

Definitions taken from \cite{Nullstellensatz}

\begin{definition}[Hilbert's Nullstellensatz]
    Given the polynomials $p_1, \ldots, p_m \in \F[x_1, \ldots, x_n]$, the equation $p_1 = \ldots = p_m = 0$ is unsolvable if and only if $\exists g_1, \ldots, g_m \in \F[x_1, \ldots, x_n]$ such that $\sum\limits_{i = 1}^m g_i p_i = 1$.
\end{definition}

Hilbert's Nullstellensatz can be used to define the following proof system:

\begin{definition}[Nullstellensatz Refutation]
    Given the set of polynomial equations $P = \{p_1 = 0, \ldots, p_m = 0\}$ over $\F[x_1, \ldots, x_n]$, where $\F$ is any field, a Nullstellensatz refutation is a set of polynomials $\pi = \{g_1, \ldots, g_n\} \subseteq \F[x_1, \ldots, x_n]$ such that $\sum\limits_{i = 1}^m g_i p_i = 1$.

    The set of polynomials $P = \{p_1, \ldots, p_n\}$ is called the axiom set and the set $\pi = \{g_1, \ldots, g_n, h_1, \ldots, h_m\}$ is called proof of $P$.
\end{definition}

By also adding the polynomial equations $x_1^2-x_1 = 0, \ldots, x_n^2-x_n = 0$ to the set of axioms, the $\NS$ proof system is sound and complete for the set of unsatisfiable CNF formulas. Thus, in general, given the set of axioms $P = \{p_1 = 0, \ldots, p_m = 0, x_1^2-x_1 = 0, \ldots, x_n^2-x_n = 0\}$, we say that $\pi = \{g_1, \ldots, g_m, h_1, \ldots, h_n\}$ is a CNF proof of $P$ if:
\[\sum_{i = 1}^m g_i p_i + \sum_{j = 1}^n h_j (x_j^2-x_j) = 1\]

For any proof $\pi = \{g_1, \ldots, g_n, h_1, \ldots, h_m\}$ of the axioms $P = \{p_1, \ldots, p_n\}$, we define the \textit{degree of $\pi$} as:
\[\deg(\pi) = \max\{\deg(g_i p_i), \deg(h_j) + 2 \mid 1 \leq i \leq n, 1 \leq j \leq m\}\]

If $P$ has a proof $\pi$ of degree $\deg(\pi) = d$ then we say that $P \vdash^\NS_d 1$.

\begin{proposition}
    \label{neg_refutation}
    Given a set of axioms $P$, if $P \vdash^\NS_d q$ then $P, 1-q \vdash^\NS_d 1$ 
\end{proposition}

\begin{proof}
    
    Since $P \vdash^\NS_d q$, we know that $\exists g_1, \ldots, g_m, h_1, \ldots, h_n \in \F[x_1, \ldots, x_n]$ such that:
    \[\sum_{i = 1}^m g_i p_i + \sum_{j = 1}^n h_j (x_j^2-x_j) = q\]
    where $\deg(q) = d$.

    Let $p_{m+1} := 1-q$ and $P' = P \cup \{p_{m+1} = 0\}$. We define $g_1', \ldots, g_m', g_{m+1}'$ as:
    \[g_i' = \left \{ \begin{array}{ll}
        1 & \text{if } i = m+1 \\
        g_i & \text{otherwise}  \\
    \end{array}\right .\]
    
    With simple algebra we get that:
    \[\sum_{i = 1}^{m+1} g_i' p_i + \sum_{j = 1}^n h_j (x_j^2-x_j) = g_{m+1}'p_{m+1} + \sum_{i = 1}^{m} g_i' p_i + \sum_{j = 1}^n h_j (x_j^2-x_j) = (1-q) + q = 1\]
    
    thus $\pi = \{g_1', \ldots, g_{m+1}', h_1, \ldots, h_n\}$ is a proof of $P$. Moreover, since $\deg(q) = d$ implies that $\deg(g_{m+1}'p_{m+1}) = d$, it's easy to see that $\deg(\pi) = d$ holds, concluding that $P, 1-q \vdash_d^\NS 1$

\end{proof}

\begin{lemma}
    \label{union_ref}
    Given two disjoint axiom sets $P_1, P_2$, if $P_1, p \vdash_{d_1}^\NS 1$ and $P_2, 1-p \vdash_{d_2}^\NS 1$ then $P_1, P_2 \vdash_{d_1+ d_2}^\NS  1$.
\end{lemma}

\begin{proof}
    Suppose that $P_1 = \{p_1, \ldots, p_m\}$ and $P_2 = \{q_1, \ldots, q_k\}$. Let $p_{m+1} = p$ and let $q_{k+1} = 1-p$. By hypothesis, we know that
    \[\sum_{i = 1}^{m+1} g_i p_i + \sum_{j = 1}^n a_j (x_j^2-x_j) = 1\]
    for some $g_1, \ldots, g_{m+1}, a_1, \ldots, a_n$, implying that:
    \[\sum_{i = 1}^{m} g_i p_i + \sum_{j = 1}^n a_j (x_j^2-x_j) = 1 - g_{m+1} p_{m+1} = 1-g_{m+1} p\]
    
    Likewise, we know that:
    \[\sum_{i = 1}^{k+1} r_i p_i + \sum_{j = 1}^n b_j (x_j^2-x_j) = 1\]
    for some $r_1, \ldots, r_{k+1}, b_1, \ldots, b_n$, implying that:
    \[\sum_{i = 1}^{k} r_i p_i + \sum_{j = 1}^n b_j (x_j^2-x_j) = 1-r_{k+1}q_{k+1} = 1-r_{k+1}(1-p)\]

    We notice that:
    \[\begin{split}
        (1-p) \left (\sum_{i = 1}^{m} g_i p_i + \sum_{j = 1}^n a_j (x_j^2-x_j) \right ) &= (1-p)(1-g_{m+1} p) \\
        &= 1-g_{m+1} p - p + g_{m+1} p^2 \\
        &= 1-p 
    \end{split}\]

    In the last step, we used the fact that, due to multilinearity, it holds that $p^2 = p$. Proceeding the same way, we find that:
    \[\begin{split}
        p\, \left (\sum_{i = 1}^{k} r_i p_i + \sum_{j = 1}^n b_j (x_j^2-x_j)  \right ) &= p\,(1-r_{k+1}(1-p)) \\
        &= p \,(1-r_{k+1} + r_{k+1}p) \\
        &= p - r_{k+1} p + r_{k+1} p^2 \\
        &= p 
    \end{split}\]

    Now, we define $s_1, \ldots, s_{m+k}$ 
    \[s_i = \left \{ \begin{array}{ll}
        g_i \cdot (1-p) & \text{if } 1 \leq i \leq m \\
        r_i \cdot p & \text{if } m+1 \leq i \leq k \\
    \end{array} \right .\]
    
    and $h_1, \ldots, h_n$ as $h_j = a_j \cdot (1-p) + b_j \cdot p$.
    
    At this point, through simple algebra we get that:
    \[\sum_{i = 1}^{m+k} s_i p_i + \sum_{j = 1}^n h_j (x_j^2-x_j) =\]
    \[(1-p) \left (\sum_{i = 1}^{m} g_i p_i + \sum_{j = 1}^n a_j (x_j^2-x_j) \right ) + p \left (\sum_{i = 1}^{k} r_i p_i + \sum_{j = 1}^n b_j (x_j^2-x_j)  \right )=\]
    \[(1-p)(1-g_{m+1}p) + p\,(1-r_{k+1}(1-p)) = p + 1- p = 1\]

    concluding that $\pi_3 = \{s_1, \ldots, s_{m+k}, h_1, \ldots, h_n\}$ is a proof of $P_1 \cup P_2$. Furthermore, we notice that:
    \[\deg((1-p)(1-g_{m+1}p)) = \deg(1-p) + \deg(1-g_{m+1}p) = d_1 + d_2\]
    and that:
    \[\deg(p\,(1-r_{k+1}(1-p))) = \deg(p) + \deg(1-r_{k+1}(1-p)) = d_2 + d_1\]

    Finally, we get that:
    \[\deg(\pi_3) = \max(\deg((1-p)(1-g_{m+1}p)), \deg(p\,(1-r_{k+1}(1-p)))) = d_1 + d_2\]
    concluding that $P_1, P_2 \vdash_{d_1+d_2}^\NS 1$.

\end{proof}

\newpage

\section{Treelike $\Res$ and Nullstellensatz}

\begin{definition}[$\FNS$ encoding of $\Res$]
    Given a $\Res$ linear clause $C = \bigvee\limits_{i = 0}^{k_1} x_i \lor  \bigvee\limits_{j = 0}^{k_2} \overline{x_j}$, the $\FNS$ encoding of $C$ is defined as $\enc(C) := \prod\limits_{i = 0}^{k_1} x_i \cdot \prod\limits_{j = 0}^{k_2} (1-x_j)$.
    
    In general, a $\ResP$ formula $F = C_1 \land \ldots \land C_m$ defined on the variables $x_1, \ldots, x_n$ gets encoded in $\FNS$ as the set of axioms $P_F = \{\enc(C_i) = 0 \mid 1 \leq i \leq m\} \cup \{x_j^2-x_j = 0 \mid 1 \leq j \leq n\}$.
\end{definition}

\begin{theorem}
    Let $F$ be an unsatisfiable CNF. If $T$ is $\ResP$ refutation of $F$ of size $s$ then there is $\NS$ refutation of $F$ of degree $O(\log(s))$.
\end{theorem}

\begin{proof}
    Let $F = C_1 \land \cdots \land C_n$. We proceed by strong induction on the size $s$.
    
    If $s = 1$ then the $T$ contains only the empty clause $\bot$, meaning that it also is one of the starting clauses and thus one of the axioms. We notice that $\mathrm{enc}(\bot) = 1$, which easily concludes that $\bot \vdash_{0}^\NS 1$.

    Suppose now that $s > 1$. Let $\mathcal{L}$ be axioms of $T$. Since $T$ is a binary tree, by \Cref{13_23_lewis} we know that there is a clause $C_k$, i.e. a node, of $T$ such that $T_{C_k}$ has size between $\floor{\frac{1}{3} s}$ and $\ceil{\frac{2}{3} s}$.

    Let $T' = (T - T_{C_k}) \cup \{C_k\}$. Due to the size of $T_{C_k}$, we get that $T'$ has size between $\floor{\frac{1}{3} s}+1$ and $\ceil{\frac{2}{3} s}+1$. Moreover, we notice that since $T$ is a treelike refutation it holds that $T_{C_k}$ and $T'$ work with different clauses (except $C_k$), thus their axioms are disjoint. Let $\mathcal{L}_1, \mathcal{L}_2$ be the two sets of axioms respectively used by $T_{C_k}$ and $T'$.
    
    By construction, we notice that $T_{C_k}$ derives the clause $C_k$ using the axioms $\mathcal{L}_1$, while $T_{C_k}$ derives the clause $\bot$ using the axioms $\mathcal{L}_2, C_k$. Thus, since $T_{C_k}$ and $T'$ have size lower than $s$, by induction hypothesis we get that $\enc(\mathcal{L}_1) \vdash_{c_1 \cdot \log s}^\NS \enc(C_k)$ and $\enc(\mathcal{L}_2), \enc(C_k) \vdash_{c_2 \cdot \log s}^\NS 1$ for some constants $c_1, c_2$. By \Cref{neg_refutation} we easily conclude that $\enc(\mathcal{L}_1), (1- \enc(C_k)) \vdash_{c_1 \cdot \log s}^\NS 1$ and, by \Cref{union_ref}, that $\enc(\mathcal{L}_1), \enc(\mathcal{L}_2) \vdash_{(c_1+c_2) \cdot \log s}^\NS 1$. Finally, since $\mathcal{L}_1 \cup \mathcal{L}_2 = \mathcal{L}$, we get that $\enc(\mathcal{L}) \vdash_{(c_1+c_2) \cdot \log s}^\NS 1$, meaning that $\mathcal{L}$ has a $\NS$ refutation of degree $O(\log s)$.

\end{proof}

\newpage

\section{Treelike $\ResP$ and Nullstellensatz}

\begin{definition}[$\FNS$ encoding of $\Res(\oplus)$]
    Given a $\ResP$ linear clause $C = \bigvee\limits_{i = 0}^k (\ell_i = \alpha_i)$, the $\FNS$ encoding of $C$ is defined as $\enc_{\oplus}(C) := \prod\limits_{i = 0}^k (\alpha - \ell_i)$.
    
    In general, a $\ResP$ formula $F = C_1 \land \ldots \land C_m$ defined on the variables $x_1, \ldots, x_n$ gets encoded in $\FNS$ as the set of axioms $P_F = \{\enc_{\oplus}(C_i) = 0 \mid 1 \leq i \leq m\} \cup \{x_j^2-x_j = 0 \mid 1 \leq j \leq n\}$.
\end{definition}

\begin{theorem}[\cite{res_parity}]
    \;
    \begin{enumerate}
        \item Every tree-like $\ResP$ proof of an unsatisfiable formula $F$ may be translated to a parity decision tree for $F$ without increasing the size of the tree.
        
        \item Every parity decision tree for an unsatisfiable linear CNF may be translated into a tree-like $\ResP$ proof and the size of the resulting proof is at most twice the size of the parity decision tree (and where the weakening is applied only to the axioms).
    \end{enumerate}
\end{theorem}

\begin{corollary}
    Every tree-like $\ResP$ proof of an unsatisfiable formula $F$ can be converted to a tree-like $\ResP$ proof of at most double the size and with weakening applied only to the axioms.
\end{corollary}


\cleardoublepage