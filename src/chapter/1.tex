\hypersetup{colorlinks=true, linkcolor=blue, citecolor=red}

\chapter{Introduction} \label{chap:introduction}

For many years, the study of \textit{decision problems} has been the main focus of computability theory. These types of problems include any problem that can be described as a simple question with a yes-or-no answer, such as asking if some input object has got some kind of property or not. Decidability theory plays a core rule in math and computer science due to the subjects studied by both fields. However, even thought decision problem can be computed by an algorithm producing a solution for a given input, not all decision problems can be solved "efficiently", meaning in a reasonable amount of time.

This very nature of decision problems has given birth to complexity theory, the field of computer science focused on solving the following fundamental question commonly referred to as the \textsf{P} vs.\@ \textsf{NP} problem:\@ \flqq does every decision problem with an efficient way to \textit{verify} a solution for an input also have an efficient way to \textit{solve} the problem for that input?\frqq. The hardness of this question sparked into the establishment of many subsets of complexity theory, in particular proof complexity, communication complexity and circuit complexity. Each of these subfields revolve around solving this question for one single \textsf{NP}-Complete problem, that being problems for which finding an efficient algorithm would automatically imply that \textsf{P} = \textsf{NP}.

In recent years, the study of decision problems has been generalized to the study of \textit{functional problems}, i.e. any problem where an output that is more complex than a yes-or-no answer is expected for a given input. By their very nature, functional problems are an "harder type" of problems respect to decision problems, describing any possible type of computation achievable through the concept of mathematical function and algorithm. In the same fashion of decision problems, the study of functional problems focuses on the \textsf{FP} vs.\@ \text{FNP} question. In particular, solving this question for the functional case would also imply solving it for the decisional case and vice versa.

The study of functional problems has given many important results through its characterization both as a \textit{black-box model} and a \textit{white-box model}. These two models have been shown to be highly correlated to the previous subfields of decision problems, in particular proof complexity. These correlations give a way to study the latter through the lens of functional problems, which enables "less restrictive" methods of study. Moreover, it has been shown that so-called \textit{natural proof systems} effectively correspond to black-box total functional problems.

Recent studies discuss if the already known relations between white-box functional problems, communication complexity and circuit complexity can be extended in order to establish a strong characterization as the one found for proof complexity. If this characterization is solid, the study of total functional problems would play an essential role in complexity theory, becoming an \textit{unifying theory} between its subfields and thus implying that any result found in any subfield would also have impact in the others.

However, such results have not yet been found, even though a lot of process has been made. In particular, the concepts of \textit{feasible interpolation} and \textit{query-to-communication lifting theorems} play a very important role in finding such relations, which however are still not enough to ensure a solid "organizing theory". Nonetheless, such unifying characterizations are well-defined for highly studied proof systems and circuit models, such as Treelike Resolution and Monotone Formula Circuits.

The objective of this work is to study the relations found in proof, communication and circuit complexity, formalizing the concepts involved in each field and giving an in-depth view on how they are linked to the study of total functional problems.

\cleardoublepage