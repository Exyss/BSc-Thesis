
\chapter{Introduction} \label{chap:introduction}

For many years, the study of \textbf{decision problems} has been the main focus of computability theory. These types of problems include any problem that can be described as a simple question with a yes-or-no answer, such as asking if some input object has got some kind of property or not. Decidability theory plays a core rule in math and computer science due to the subjects studied by both fields. However, not all decision problems can be solved efficiently by an algorithm, i.e. be decided in a reasonable amount of finite steps.

This very nature of decision problems has given birth to \textbf{complexity theory}, the field of computer science focused on solving the following fundamental question commonly referred to as the \textsf{P} vs.\@ \textsf{NP} question: \say{does every decision problem with an efficient way to \textit{verify} a solution for an input also have an efficient way to \textit{solve} the problem for that input?}. The hardness of this question sparked into the establishment of many subsets of complexity theory, in particular proof complexity, communication complexity and circuit complexity. Each of these subfields revolve around solving this question for one single \textsf{NP}-Complete problem, that being a problem for which finding an efficient algorithm would automatically imply that \textsf{P} = \textsf{NP}.

In recent years, the study of decision problems has been generalized into the study of \textbf{functional problems}, i.e. any problem where an output that is more complex than a yes-or-no answer is expected for a given input. By their very nature, functional problems are an "harder type" of problems respect to decision problems, describing any possible type of computation achievable through the concept of mathematical function and algorithm. In the same fashion of decision problems, the study of functional problems focuses on the \textsf{FP} vs.\@ \textsf{FNP} question. In particular, solving this question for the functional case would also imply solving it for the decisional case and vice versa.

The study of functional problems has given many important results through its characterization both as a \textbf{black-box model} and a \textbf{white-box model}. These two models have been shown to be highly correlated to the subfields of complexity theory afore mentioned. These correlations give a way to study these subfields through the lens of total functional problems, making it an \textit{unifying theory}, even though currently this statement seems to be farfetched for the white-box model.

\change{Add stuff on parity here}

\cleardoublepage