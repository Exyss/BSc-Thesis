\chapter*{Conclusions}

\addcontentsline{toc}{chapter}{Conclusions}

The relations between total search problems, protocols, circuits and proofs make \textsf{TFNP} an interesting theory that may be even capable of capturing all of these branches of complexity theory under an \textit{universal theory}. However, such universal characterizations are still \textit{fuzzy}: relations between the white-box and black-box models such as \textit{query-to-communication lifting} theorems and \textit{interpolation theorems} are not strong enough.

We have shown that parity applied to black-box total search problems defines a stronger computational model through parity decision trees. Moreover, we have discussed how parity decision trees are easily characterized by Tree-like Linear Resolution over $\F_2$. Through this characterization, we were able to show that this model's computational power seems to be limited when applied in this context.

\begin{figure}[H]
    \centering
    
    \begin{tikzpicture}[->,>=stealth,shorten >=1pt,auto,node distance=1.25cm, thick,main node/.style={scale=0.9,circle,draw,font=\sffamily\normalsize}]
    
        \node[rectangle, draw, rounded corners, minimum width=15mm, minimum height=6mm]  (1) []{\textsf{FNP}};

        \node[rectangle, draw, rounded corners, minimum width=15mm, minimum height=6mm]  (2) [below of = 1]{\textsf{TFNP}$^{dt}$};

        \node[rectangle, draw, rounded corners, minimum width=15mm, minimum height=6mm]  (3) [below of = 2]{\textsf{PPP}$^{dt}$};

        \node[rectangle, draw, rounded corners, minimum width=15mm, minimum height=6mm]  (4) [left of = 3, xshift=-30]{\textsf{PLS}$^{dt}$};

        \node[rectangle, draw, rounded corners, minimum width=15mm, minimum height=6mm]  (5) [right of = 3, xshift=30]{\textsf{PPA}$^{dt}$};

        \node[rectangle, draw, rounded corners, minimum width=15mm, minimum height=6mm]  (6) [below of = 3]{\textsf{PPADS}$^{dt}$};

        \node (7) [below of = 6]{};

        \node[rectangle, draw, rounded corners, minimum width=15mm, minimum height=6mm]  (8) [left of = 7, xshift=-30]{\textsf{SOPL}$^{dt}$};

        \node[rectangle, draw, rounded corners, minimum width=15mm, minimum height=6mm]  (9) [right of = 7, xshift=30]{\textsf{PPAD}$^{dt}$};

        \node[rectangle, draw, rounded corners, minimum width=15mm, minimum height=6mm]  (10) [below of = 7]{\textsf{CLS}$^{dt}$};

        \node (13) [right of = 10, xshift=30]{};

        \node[rectangle, draw, rounded corners, minimum width=15mm, minimum height=6mm]  (11) [below of = 13]{\textsf{FP}$^{dt}$};
    
        \node[rectangle, draw, rounded corners, minimum width=15mm, minimum height=6mm, color=blue]  (12) [right of = 9, xshift=100]{\textsf{FP}$^{pdt}$};

        \path[every node/.style={font=\sffamily\small}]
            (2) edge (1)
            (3) edge (2)
            (4) edge (2)
            (5) edge (2)
            (6) edge (3)
            (8) edge (4)
            (8) edge (6)
            (9) edge (5)
            (9) edge (6)
            (10) edge (8)
            (10) edge (9)
            (11) edge (10)

            (4) edge[dashed, near end, color=blue] node[yshift=10, xshift=-40]{\Cref{pls_not_inside_fp_pdt}}(12) 
            (11) edge[color=blue] node{\Cref{fp_dt_inside_fp_pdt}}(12) 
            (12) edge[bend right, swap, color=blue] node{\Cref{main_thm}}(5) 
            (12) edge[dashed, bend left, color=blue] node{\Cref{fp_pdt_not_inside_fp_dt}}(11) 
            ;
    \end{tikzpicture}

    
    \caption{$\TFNPdt$ hierarchy extended through our results. An arrow $A \to B$ means that $A \subseteq B$ while a dashed arrow $A \dashrightarrow B$ means that $A \not\subseteq B$.}   
\end{figure} 

\newpage

\newpage

The results open questions that further research could investigate:
\begin{enumerate}
    \item We have discussed how parity decision trees are stronger than decision trees. Can the whole black-box \textsf{TFNP} hierarchy also be modeled through PDTs? Is $\mathsf{TFNP}^{pdt}$ a stronger model? How is it related to $\mathsf{TFNP}^{dt}$?
    \item Reductions in the current black-box model are modeled through decision trees. Can these reductions be modeled through parity decision tree reductions? Do the same inclusions between classes of the hierarchy hold?
    \item Our results are based on the $\F_2$ field. Are they extendable to $\F_n$? Is Tree-like Linear Resolution over $\F_n$ capable of characterizing a computational model that generalizes parity decision trees?
    \item Parity decision trees are also related to communication complexity \cite{parity_complexity, pdts_comm_compl}. Is there an equivalent model also in white-box \textsf{TFNP}? Is it characterized by a specific type of circuit model?
\end{enumerate}